\documentclass[article]{jss}
\usepackage[utf8]{inputenc}

\providecommand{\tightlist}{%
  \setlength{\itemsep}{0pt}\setlength{\parskip}{0pt}}

\author{
John Paul Helveston\\George Washington University
}
\title{Flexible Multinomial Logit Models with Preference Space and
Willingness-to-Pay Space Utility Specifications in R: The \pkg{logitr}
Package}

\Plainauthor{John Paul Helveston}
\Plaintitle{Flexible Multinomial Logit Models with Preference Space and
Willingness-to-Pay Space Utility Specifications in R: The logitr Package}
\Shorttitle{\pkg{logitr}: Preference and WTP Space Multinomial Logit Models}

\Abstract{
In many applications of discrete choice models, modelers are interested
in estimating consumer's marginal ``willingness-to-pay'' (WTP) for
different attributes. WTP can computed by dividing the estimated
parameters of a utility model in the preference space by the price
parameter or by estimating a utility model in the WTP space. For
homogeneous models, these two procedures generally produce the same
estimates of WTP, but the same is not true for heterogeneous models
where model parameters are assumed to follow a specific distribution.
The \pkg{logitr} package was written to allow for flexible estimation of
multinomial logit models with preference space and WTP space utility
specifications. The package supports homogeneous multinomial logit (MNL)
and heterogeneous mixed logit (MXL) models, including support for normal
and log-normal parameter distributions. Since MXL models and models with
WTP space utility specifications are non-convex, an option is included
to run a multi-start optimization loop with random starting points in
each iteration. The package also includes a market simulation function
to estimate the expected market shares of a set of alternatives using an
estimated model.
}

\Keywords{multinomial logit, preference space, willingness-to-pay space, discrete choice, \proglang{R}}
\Plainkeywords{multinomial logit, preference space, willingness-to-pay space, discrete choice, R}

%% publication information
%% \Volume{50}
%% \Issue{9}
%% \Month{June}
%% \Year{2012}
%% \Submitdate{}
%% \Acceptdate{2012-06-04}

\Address{
    John Paul Helveston\\
  George Washington University\\
  Science \& Engineering Hall\hfill\break 800 22nd St
  NW\hfill\break Washington, DC 20052\\
  E-mail: \email{jph@gwu.edu}\\
  URL: \url{http://jhelvy.com}\\~\\
  }

% Pandoc header

\usepackage{amsmath} \usepackage{upgreek}

\begin{document}

\newcommand{\betaVec}{\boldsymbol\upbeta}
\newcommand{\omegaVec}{\boldsymbol\upomega}
\newcommand{\zetaVec}{\boldsymbol\upzeta}
\newcommand{\deltaVec}{\boldsymbol\updelta}
\newcommand{\gammaVec}{\boldsymbol\upgamma}
\newcommand{\epsilonVec}{\boldsymbol\upepsilon}
\newcommand{\xVec}{\mathrm{\mathbf{x}}}
\newcommand{\XVec}{\mathrm{\mathbf{X}}}

\hypertarget{introduction}{%
\section{Introduction}\label{introduction}}

In many applications of discrete choice models, modelers are interested
in estimating consumer's marginal ``willingness-to-pay'' (WTP) for
different attributes. WTP can be estimated in two ways:

\begin{enumerate}
\def\labelenumi{\arabic{enumi}.}
\tightlist
\item
  Estimate a discrete choice model in the ``preference space'' where
  parameters have units of utility and then compute the WTP by dividing
  the parameters by the price parameter.
\item
  Estimate a discrete choice model in the ``WTP space'' where parameters
  have units of WTP.
\end{enumerate}

While the two procedures generally produce the same estimates of WTP for
homogenous models, the same is not true for heterogeneous models where
model parameters are assumed to follow a specific distribution, such as
normal or log-normal \citep{Train2005}. For example, in a preference
space specification, a normally distributed attribute parameter divided
by a log-normally distributed price parameter produces a strange WTP
distribution with large tails. In contrast, a WTP space specification
allows the modeler to directly assume WTP is normally distributed. The
\pkg{logitr} package was developed to enable modelers to choose between
these two utility spaces when estimating multinomial logit models.

\hypertarget{the-random-utility-model}{%
\section{The random utility model}\label{the-random-utility-model}}

The random utility model is a well-established framework in many fields
for estimating consumer preferences from observed consumer choices
\citep[\citet{Train2009}]{Louviere2000}. Random utility models assume
that consumers choose the alternative \(j\) a set of alternatives that
has the greatest utility \(u_{j}\). Utility is a random variable that is
modeled as \(u_{j} = v_{j} + \varepsilon_{j}\), where \(v_{j}\) is the
``observed utility'' (a function of the observed attributes such that
\(v_{j} = f(\mathrm{\mathbf{x}}_{j})\)) and \(\varepsilon_{j}\) is a
random variable representing the portion of utility unobservable to the
modeler.

Adopting the same notation as in \citep{Helveston2018}, consider the
following utility model:

\input{./eqns/utility.Rmd}

where \(\boldsymbol\upbeta^{*}\) is the vector of coefficients for
non-price attributes \(\mathrm{\mathbf{x}}_{j}\), \(\alpha^{*}\) is the
coefficient for price \(p_{j}\), and the error term,
\(\varepsilon^{*}_{j}\), is an IID random variable with a Gumbel extreme
value distribution of mean zero and variance \(\sigma^2(\pi^2/6)\). This
model is not identified since there exists an infinite set of
combinations of values for \(\boldsymbol\upbeta^{*}\), \(\alpha^{*}\),
and \(\sigma\) that produce the same choice probabilities. In order to
specify an identifiable model, the modeler must normalize equation
(\ref{eq:utility}). One approach is to normalize the scale of the error
term by dividing equation (\ref{eq:utility}) by \(\sigma\), producing
the ``preference space'' utility specification:

\input{./eqns/utilityPreferenceScaled.Rmd}

The typical preference space parameterization of the multinomial logit
(MNL) model can then be written by rewriting equation
(\ref{eq:utilityPreferenceScaled}) with \(u_j = (u^*_j / \sigma)\),
\(\boldsymbol\upbeta= (\boldsymbol\upbeta^{*} / \sigma)\),
\(\alpha = (\alpha^{*} / \sigma)\), and
\(\varepsilon_{j} = (\varepsilon^{*}_{j} / \sigma)\):

\input{./eqns/utilityPreference.Rmd}

The vector \(\boldsymbol\upbeta\) represents the marginal utility for
changes in each non-price attribute, and \(\alpha\) represents the
marginal utility obtained from price reductions. In addition, the
coefficients \(\boldsymbol\upbeta\) and \(\alpha\) are measured in units
of \emph{utility}, which only has relative rather than absolute meaning.

The alternative normalization approach is to normalize equation
(\ref{eq:utility}) by \(\alpha^*\) instead of \(\sigma\), producing the
``willingness-to-pay (WTP) space'' utility specification:

\input{./eqns/utilityWtpScaled.Rmd}

Since the error term in equation is scaled by
\(\lambda^2 = \sigma^2/(\alpha^{*})^2\), we can rewrite equation
(\ref{eq:utilityWtpScaled}) by multiplying both sides by
\(\lambda= (\alpha^{*} / \sigma\)) and renaming
\(u_j = (\lambda u^*_j / \alpha^*)\),
\(\boldsymbol\upomega= (\boldsymbol\upbeta^{*} / \alpha^{*}\)), and
\(\varepsilon_j = (\lambda \varepsilon^*_j / \alpha^*)\):

\input{./eqns/utilityWtp.Rmd}

Here \(\boldsymbol\upomega\) represents the marginal WTP for changes in
each non-price attribute, and \(\lambda\) represents the scale of the
deterministic portion of utility relative to the standardized scale of
the random error term.

The utility models in equations \ref{eq:utilityPreference} and
\ref{eq:utilityWtp} represent the preference space and WTP space utility
specifications, respectively. In equation \ref{eq:utilityPreference},
WTP is estimated as \(\hat{\boldsymbol\upbeta} / \hat{\alpha}\); in
equation \ref{eq:utilityWtp}, WTP is simply
\(\hat{\boldsymbol\upomega}\).

\hypertarget{the-logitr-package}{%
\section{The logitr package}\label{the-logitr-package}}

\hypertarget{installation}{%
\subsection{Installation}\label{installation}}

First, make sure you have the \pkg{devtools} library installed:

\begin{CodeChunk}

\begin{CodeInput}
R> install.packages("devtools")
\end{CodeInput}
\end{CodeChunk}

Then load the \pkg{devtools} library and install the \pkg{logitr}
package:

\textbackslash{}begin\{CodeChunk\}

\textbackslash{}begin\{CodeInput\} R\textgreater{} library(``devtools'')
R\textgreater{} install\_github(``jhelvy/logitr'') R\textgreater{}
library(``logitr'') \textbackslash{}end\{CodeInput\}
\textbackslash{}end\{CodeChunk\}

\hypertarget{data-format}{%
\subsection{Data format}\label{data-format}}

\bibliography{library.bib}


\end{document}

