\documentclass[article]{jss}
\usepackage[utf8]{inputenc}

\providecommand{\tightlist}{%
  \setlength{\itemsep}{0pt}\setlength{\parskip}{0pt}}

\author{
John Paul Helveston\\George Washington University
}
\title{Multinomial Logit Models with Preference Space and Willingness-to-Pay
Space Utility Specifications in R: The \pkg{logitr} Package}

\Plainauthor{John Paul Helveston}
\Plaintitle{Multinomial Logit Models with Preference Space and Willingness-to-Pay
Space Utility Specifications in R: The logitr Package}
\Shorttitle{\pkg{logitr}: Preference and WTP Space Multinomial Logit Models}

\Abstract{
The abstract of the article.
}

\Keywords{multinomial logit, preference space, willingness-to-pay space, discrete choice, \proglang{R}}
\Plainkeywords{multinomial logit, preference space, willingness-to-pay space, discrete choice, R}

%% publication information
%% \Volume{50}
%% \Issue{9}
%% \Month{June}
%% \Year{2012}
%% \Submitdate{}
%% \Acceptdate{2012-06-04}

\Address{
    John Paul Helveston\\
  George Washington University\\
  Science \& Engineering Hall 800 22nd St NW Washington, DC 20052\\
  E-mail: \email{john.helveston@gmail.com}\\
  URL: \url{http://jhelvy.com}\\~\\
  }

% Pandoc header

\usepackage{amsmath} \usepackage{upgreek}

\begin{document}

\newcommand{\betaVec}{\boldsymbol\upbeta}
\newcommand{\omegaVec}{\boldsymbol\upomega}
\newcommand{\zetaVec}{\boldsymbol\upzeta}
\newcommand{\deltaVec}{\boldsymbol\updelta}
\newcommand{\gammaVec}{\boldsymbol\upgamma}
\newcommand{\epsilonVec}{\boldsymbol\upepsilon}
\newcommand{\xVec}{\mathrm{\mathbf{x}}}
\newcommand{\XVec}{\mathrm{\mathbf{X}}}

\hypertarget{introduction}{%
\section{Introduction}\label{introduction}}

In many applications of discrete choice models, modelers are interested
in estimating consumer's marginal ``willingness-to-pay'' (WTP) for
different attributes (cites).

\hypertarget{the-random-utility-model}{%
\section{The Random Utility Model}\label{the-random-utility-model}}

The random utility model is a well-established framework in many fields
for estimating consumer preferences from observed consumer choices
\citep[\citet{Train2009}]{Louviere2000}. Random utility models assume
that consumers choose the alternative \(j\) a set of alternatives that
has the greatest utility \(u_{j}\). Utility is a random variable that is
modeled as \(u_{j} = v_{j} + \varepsilon_{j}\), where \(v_{j}\) is the
``observed utility'' (a function of the observed attributes such that
\(v_{j} = f(\mathrm{\mathbf{x}}_{j})\)) and \(\varepsilon_{j}\) is a
random variable representing the portion of utility unobservable to the
modeler.

Consider the following utility model:

\input{./eqns/utility.Rmd}

where \(\boldsymbol\upbeta^{*}\) is the vector of coefficients for
non-price attributes \(\mathrm{\mathbf{x}}_{j}\), \(\alpha^{*}\) is the
coefficient for price \(p_{j}\), and the error term,
\(\varepsilon^{*}_{j}\), is an IID random variable with a Gumbel extreme
value distribution of mean zero and variance \(\sigma^2(\pi^2/6)\). This
model is not identified since there exists an infinite set of
combinations of values for \(\boldsymbol\upbeta^{*}\), \(\alpha^{*}\),
and \(\sigma\) that produce the same choice probabilities. In order to
specify an identifiable model, the modeler must normalize equation
(\ref{eq:utility}). One approach is to normalize the scale of the error
term by dividing equation (\ref{eq:utility}) by \(\sigma\), producing
the ``preference space'' utility specification \citep{Train2005}:

\input{./eqns/utilityPreferenceScaled.Rmd}

The typical preference space parameterization of the multinomial logit
(MNL) model can then be written by rewriting equation
(\ref{eq:utilityPreferenceScaled}) with \(u_j = (u^*_j / \sigma)\),
\(\boldsymbol\upbeta= (\boldsymbol\upbeta^{*} / \sigma)\),
\(\alpha = (\alpha^{*} / \sigma)\), and
\(\varepsilon_{j} = (\varepsilon^{*}_{j} / \sigma)\):

\input{./eqns/utilityPreference.Rmd}

The vector \(\boldsymbol\upbeta\) represents the marginal utility for
changes in each non-price attribute, and \(\alpha\) represents the
marginal utility obtained from price reductions. In addition, the
coefficients \(\boldsymbol\upbeta\) and \(\alpha\) are measured in units
of \emph{utility}, which only has relative rather than absolute meaning.

The alternative normalization approach is to normalize equation
(\ref{eq:utility}) by \(\alpha^*\) instead of \(\sigma\), producing the
``willingness-to-pay (WTP) space'' utility specification
\citep{Train2005}:

\input{./eqns/utilityWtpScaled.Rmd}

Since the error term in equation is scaled by
\(\lambda^2 = \sigma^2/(\alpha^{*})^2\), we can rewrite equation
(\ref{eq:utilityWtpScaled}) by multiplying both sides by
\(\lambda= (\alpha^{*} / \sigma\)) and renaming
\(u_j = (\lambda u^*_j / \alpha^*)\),
\(\boldsymbol\upomega= (\boldsymbol\upbeta^{*} / \alpha^{*}\)), and
\(\varepsilon_j = (\lambda \varepsilon^*_j / \alpha^*)\):

\input{./eqns/utilityWtp.Rmd}

Here \(\boldsymbol\upomega\) represents the marginal WTP for changes in
each non-price attribute, and \(\lambda\) represents the scale of the
deterministic portion of utility relative to the standardized scale of
the random error term.

\bibliography{library.bib}


\end{document}

